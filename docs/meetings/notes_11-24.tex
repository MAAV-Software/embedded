\documentclass{article}
\usepackage{enumerate, amsfonts, amsmath, amssymb, amsthm}
\usepackage[margin=1in, top=.5in]{geometry}

\begin{document}

\title{Controls Subteam Meeting Notes}
\author{Sajan Patel}
\date{24 November 2015}
\maketitle

\section{Admin Updates}
\begin{enumerate}
\item 2 people still need to complete Basic 2 and get access.
\item Attendance: 8
\end{enumerate}

\section{Subteam Updates}
Vehicle Class Test Cases - Lixing and Ashish
\begin{enumerate}
    \item What is your deliverable for this week?

    Make test case for the setGains and setSetpt functions.

    \item Regarding that deliverable, what have you accomplished and how did
        you accomplish it?

    No. Still working on getting it to compile.

    \item What are you stuck on?

    Vehicle default ctor doesn't exist.

    \item What will you work on for the next week?
    
    Continuing with making it compile and run the simple tests.

    \item What is the high-level design you are thinking of for next week's
        deliverable?
    
    Start using ekf scripts to generate answers to test cases.

\end{enumerate}


\noindent{}EKF Class Test Cases - Nathan, Nick, Anthony
\begin{enumerate}
    \item What is your deliverable for this week?

    Complete set of test cases for ekf.

    \item Regarding that deliverable, what have you accomplished and how did
        you accomplish it?

    Parts are done. Modified Matlab script for generating test case answers 
    and continuing to learn and understand the 
    EKF math. Also learned boost unit test framework and working on setting up 
    unit tests.

    \item What are you stuck on?

    EKF math and theory.

    \item What will you work on for the next week?
    
    Four compiled and running unit test cases.

    \item What is the high-level design you are thinking of for next week's
        deliverable?
    
    Getting basic and some edge test cases designed.
\end{enumerate}


\noindent{}EKF Simulation - Sasawat
\begin{enumerate}
    \item What is your deliverable for this week?

    Visualize plots of C++ version of ekf running on log data.

    \item Regarding that deliverable, what have you accomplished and how did
        you accomplish it?

    Made program that simulates calls to ekf algorithm running on log data.

    \item What are you stuck on?

    Program shows output that is different from spKalman Matlab script.

    \item What will you work on for the next week?
    
    Continuing the simulation and find the differnces between Matlab and C++
    (flying on vehicle) versions of EKF.

    \item What is the high-level design you are thinking of for next week's
        deliverable?
    
    N/A. Focussing on implementation details.
\end{enumerate}


\noindent{}EKF Documentation - Steven 
\begin{enumerate}
    \item What is your deliverable for this week?

    Completed documentation.

    \item Regarding that deliverable, what have you accomplished and how did
        you accomplish it?

    Part of it is done following guidelines in vehicle.tex. 

    \item What are you stuck on?

    Need guidance on how the math works.

    \item What will you work on for the next week?
    
    Continuing to write up math in vehicle.tex.

    \item What is the high-level design you are thinking of for next week's
        deliverable?
    
        Making clear English explanations about the math and algorithm.
\end{enumerate}


\noindent{}Sensor Documentation - Sajan
\begin{enumerate}
    \item What is your deliverable for this week?
    
    Finding the location of pre-existing sensor noise data and sensor datasheets.

    \item Regarding that deliverable, what have you accomplished and how did
        you accomplish it?

    Found the redmine issues and emails that the data was saved in.

    \item What are you stuck on?

    Nothing.

    \item What will you work on for the next week?
    
    Moving pre-existing documentation on sensor noise to vehicle.tex in infra repo.

    \item What is the high-level design you are thinking of for next week's
        deliverable?

        N/A. It's just documentation.

\end{enumerate}

\end{document}
