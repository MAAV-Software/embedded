\documentclass{article}

\usepackage{graphicx}
\usepackage{hyperref}
\usepackage{mathtools}

\setcounter{tocdepth}{2}

\begin{document}

\title{MAAV Kalman Filter Documentation}
\author{MAAV}
\date{}
\maketitle

\section{Variables}

\begin{table}[h]
    \begin{tabular}{ll}
	    x & final vehicle state\\
	    df(x,u) & TODO: what exactly is this?\\
	    dt & change in time between iterations\\
	    P & final covariance\\
	    A & mapping of state to P\\
	    Q & covariance of the state\\
	    $L_1$ & gain TODO: more detail (without camera data)\\
	    $L_2$ & gain "               " (with camera data)\\
	    $y_1$ & sensor inputs (without camera data)\\
	    $y_2$ & sensor inputs (with camera data)\\
	    $R_1$ & covariance of the sensor data (without camera data)\\
	    $R_2$ & covariance of the sensor data (with camera data)\\
	    $c_1(x)$ & mapping that extracts the states for y (without camera data)\\
	    $c_2(x)$ & mapping that extracts the states for y (with camera data)\\
	\end{tabular}
\end{table}

% Here's the matrices.
\[
	x=
	\left[ {\begin{array}{cccccc}
				x\\
				y\\
				z\\
				\dot x\\
				\dot y\\
				\dot z\\
	\end{array} } \right]
\]

\[
	A= 
	\left[ {\begin{array}{cccccc}
				0 & 0 & 0 & 1 & 0 & 0\\
				0 & 0 & 0 & 0 & 1 & 0\\
				0 & 0 & 0 & 0 & 0 & 1\\
				0 & 0 & 0 & 0 & 0 & 0\\
				0 & 0 & 0 & 0 & 0 & 0\\
				0 & 0 & 0 & 0 & 0 & 0\\
	\end{array} } \right]
\]

\[
	Q= 
	\left[ {\begin{array}{cccccc}
				Q_{11} & 0 & 0 & 0 & 0 & 0\\
				0 & Q_{22} & 0 & 0 & 0 & 0\\
				0 & 0 & Q_{33} & 0 & 0 & 0\\
				0 & 0 & 0 & Q_{44} & 0 & 0\\
				0 & 0 & 0 & 0 & Q_{55} & 0\\
				0 & 0 & 0 & 0 & 0 & Q_{66}\\
	\end{array} } \right]
\]

\[
	c_1(x_t)=
	\left[ {\begin{array}{ccc}
				x_3\\
				x_4\\
				x_5\\
	\end{array} } \right]_t
\]

\[
	c_2(x_t)=
	\left[ {\begin{array}{ccccc}
				x_1\\
				x_2\\
				x_3\\
				x_4\\
				x_5\\
	\end{array} } \right]_t
\]

\[
	R_1=
	\left[ {\begin{array}{ccc}
				R_{11} & 0 & 0\\
				0 & R_{22} & 0\\
				0 & 0 & R_{33}\\
	\end{array} } \right]
\]

\[
	R_2=
	\left[ {\begin{array}{ccccc}
				R_{11} & 0 & 0 & 0 & 0\\
				0 & R_{22} & 0 & 0 & 0\\
				0 & 0 & R_{33} & 0 & 0\\
				0 & 0 & 0 & R_{44} & 0\\
				0 & 0 & 0 & 0 & R_{55}\\
	\end{array} } \right]
\]



\[
	df(x,u)=
	\left[ {\begin{array}{cccccc}
				x_4\\
				x_5\\
				x_6\\
				u_x\\
				u_y\\
				u_z\\
	\end{array} } \right]
\]

\[
	y_{1_t}=
	\left[ {\begin{array}{ccc}
				z_{lidar}\\
				\dot x_{px4}\\
				\dot y_{px4}\\
	\end{array} } \right]_t
\]

\[
	y_{2_t}=
	\left[ {\begin{array}{ccccc}
				x_c\\
				y_c\\
				z_{lidar}\\
				\dot x_{px4}\\
				\dot y_{px4}\\
	\end{array} } \right]_t
\]

\section{Prediction}
The prediction estimates the current state by taking the most recently measured state and adding the TODO:what exactly is f(x,u)? multiplied by the elapsed time).
\begin{align}
	x_{t+1} = x_t + dt * df(x,u)
\end{align}
The prediction does a similar estimation for the current covariance. TODO: find out more about this
\begin{align}
	P_{t+1} = P_t + dt * (AP + PA^T + Q)
\end{align}

\section{Correction Step}
When current sensor data is available, the correction step sets the current state.
\begin{align}
	L_1 = P_tC_1^T(R + C_1P_tC_1^T)^{-1}
\end{align}

\begin{align}
	X_{t+1} = X_t + L_1(y_t-c_1(x_2))
\end{align}

\begin{align}
	P_{t+1} = (I-L_1C_1)P_t
\end{align}

\end{document}
