\documentclass[]{article}
\usepackage[cm]{fullpage}
%opening
\title{CCS Instructions for MAAV Project}

\begin{document}

\maketitle

\begin{abstract}

The MAAV Project makes extensive use of linked resources to minimize the amount of change required to build it from computer to computer, especially because the MAAV Project depends on many external resources like TivaWare and CMSIS, which may not be installed in the same location across all machines. This document details the Linked Resources that require changes. 

\end{abstract}

\section{Accessing the Properties Screen}

In Code Composer Studio, in the Project Explorer pane (by default docked to the left of the screen), right click on the "maav" project and select "Properties". A window entitled "Properties for maav" should open. 

\section{Changes required in the Resource $\rightarrow$ Linked Resources Section}

In the left hand sidebar, navigate to "Resource" $\rightarrow$ "Linked Resources". In the "Linked Resources" screen, open the "Path Variables" tab. To change a path variable, double click the path variable. Change:
\begin{itemize}
\item REPO\_LOC to the directory where the repository (the "ctrl" folder) is cloned. For example C:/Sasawat/workspace/ctrl. 
\item TIVAWARE\_INSTALL to the root directory of your TivaWare install. For example C:/ti/TivaWare\_C\_Series. This directory should contain the subdirectories /utils and /driverlib. 
\item CMSIS\_LOC to the root directory of your CMSIS install. For example C:/Sasawat/CMSIS/CMSIS. This folder should contain the subdirectories /DPS\_Lib and /Include. 
\end{itemize}

\section{Changes required in the CCS Build Section}

In the left hand sidebar, navigate to "CCS Build". In the "CCS Build" screen, open the "Variables" tab. Double click the TIVAWARE\_INSTALL variable and change it to the root directory of your TivaWare install. For example C:/ti/TivaWare\_C\_Series. This directory should contain the subdirectories /utils and /driverlib. 

\section{On Linked Source Folders}

From the main CCS screen, go to the Project Explorer pane, under the maav project, delete the include, src, and sdlib linked folders if they exist. CCS should note that "Link target will remain unchanged". This operation only deletes the possibly faulty link to the resources. \\

Right click on the maav project in Project Explorer pane, and go to Import, then Filesystem. Browse through the src folder in the repo. On the right pane of the Import screen, check the src box, in the expanded list of subfolders, uncheck sdlib. In Advanced, check Create links in Workspace and choose relative to REPO\_LOC.

Right click on the maav project in Project Explorer pane, and go to Import, then Filesystem. Browse through the src/sdlib folder in the repo. Oon the right pane of the Import screen, check the sdlib box, in the expanded list of subfolders, uncheck the option box. In Advanced, check Create links in Workspace and choose relative to REPO\_LOC.

\end{document}
