\documentclass[11pt]{article}

\usepackage[top=1.0in, bottom=1.0in, left=1.0in, right=1.0in]{geometry} % This package allows you to change the border. 

\usepackage{amsmath} % Math packages.
\usepackage{amssymb}
\usepackage{amsthm}
\usepackage{bm}
\usepackage{mathrsfs}

\usepackage{times} % To make times font.


\title{ Bare-Bones Latex Template }

\author{Prof.\ James Richard Forbes\thanks{Assistant Professor of Aerospace Engineering, University of Michigan}}

\date{\today}

\begin{document}

\maketitle

\section{Introduction}

This is a ``bare-bones" latex template. I will demonstrate how to do the simplest but most useful things. 

\subsection{Equations With Numbers}

To start, how do I write an equation? Consider the equation
%
\begin{equation}
	y = m x + b .
	\label{eq:line}
\end{equation}
%
Notice two things. First, notice there's an equation number. Second, notice the variable name ``{eq:line}" under the equation and the command ``label". I can use that variable name to then reference Eq.~\eqref{eq:line}.

\subsection{Equations Without Numbers}
\label{sec:eqs_without_nums}

Sometimes you don't want to number an equation, so you'd write
%
\begin{displaymath}
	\textbf{A} \textbf{x} = \textbf{b} ,
\end{displaymath}
%
where I used ``displaymath" rather than ``equation" in the code. I can't reference this equation because it does not have a number, which is why I don't give it a variable name 

\section{Matrices and Arrays}
Sometimes you need to have an equation array:
\begin{eqnarray}
	\left[
	\begin{array}{cc}
		a & b \\
		c & d
	\end{array}
	\right]
	& = &
	\left[
	\begin{array}{c}
		x \\
		y
	\end{array}
	\right] \\
	& = &
	\left[
	\begin{array}{c}
		a x + b y \\
		c x + d y
	\end{array}
	\right] .
\end{eqnarray}
%
Notice how there's two more numbers. Say I didn't want them, then I'd write 
%
\begin{eqnarray*}
	\left[
	\begin{array}{cc}
		a & b \\
		c & d
	\end{array}
	\right]
	& = &
	\left[
	\begin{array}{c}
		x \\
		y
	\end{array}
	\right] \\
	& = &
	\left[
	\begin{array}{c}
		a x + b y \\
		c x + d y
	\end{array}
	\right] .
\end{eqnarray*}
%

Say I wanted one number and a label, but not both. Then I'd write
%
\begin{eqnarray}
	\left[
	\begin{array}{cc}
		a & b \\
		c & d
	\end{array}
	\right]
	& = &
	\left[
	\begin{array}{c}
		x \\
		y
	\end{array}
	\right] \label{eq:matrix_math} \\
	& = &
	\left[
	\begin{array}{c}
		a x + b y \\
		c x + d y
	\end{array}
	\right] , \nonumber
\end{eqnarray}
%
and now I can reference \eqref{eq:matrix_math}.

Notice I can use ``label" to give a whole section a variable name, so I can then refer to the section, such as Section \eqref{sec:eqs_without_nums}.

Also, note that you have to compile the latex code twice before your equation or section number will appear. When compiling a bibliography using bibtex you must compile as follows: ``latex bibtex latex latex". This has to do with latex creating a list and then reading the list. 

\section{Closing Remarks}

Latex is awesome. I hope you use it!

\end{document}