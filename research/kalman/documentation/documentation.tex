\documentclass[12pt]{article}

% This first part of the file is called the PREAMBLE. It includes
% customizations and command definitions. The preamble is everything
% between \documentclass and \begin{document}.

\usepackage[margin=1in]{geometry}  % set the margins to 1in on all sides
\usepackage{graphicx}              % to include figures
\usepackage{amsmath}               % great math stuff
\usepackage{amsfonts}              % for blackboard bold, etc
\usepackage{amsthm}                % better theorem environments


% various theorems, numbered by section

\newtheorem{thm}{Theorem}[section]
\newtheorem{lem}[thm]{Lemma}
\newtheorem{prop}[thm]{Proposition}
\newtheorem{cor}[thm]{Corollary}
\newtheorem{conj}[thm]{Conjecture}

\DeclareMathOperator{\id}{id}

\newcommand{\bd}[1]{\mathbf{#1}}  % for bolding symbols
\newcommand{\RR}{\mathbb{R}}      % for Real numbers
\newcommand{\ZZ}{\mathbb{Z}}      % for Integers
\newcommand{\col}[1]{\left[\begin{matrix} #1 \end{matrix} \right]}
\newcommand{\comb}[2]{\binom{#1^2 + #2^2}{#1+#2}}


\begin{document}


\title{Continuous Discrete Extended Kalman Filter Documentation}

\author{Zhengjie Cui, Greg Miller, Clark Zhang \\ 
Michigan Autonomous Aeriel Vehicles \\
University Of Michigan, Ann Arbor}

\maketitle
\begin{abstract}
This contains documentation about research and implementation of our state estimation filter used for the MAAV vehicle 2015-2016.
\end{abstract}

\section{Basic Sensors and definitions overview}
For sensors, we have
\begin{enumerate}
\item{
Px4Flow Camera that gives us (the following are in the frame of reference of the camera and not global frame)
\begin{enumerate}
\item $\frac{dx}{dt}$
\item $\frac{dy}{dt}$
\item $z$
\end{enumerate}
}
\item{
Microstrain IMU
\begin{enumerate}
\item roll
\item pitch
\item yaw
\end{enumerate}
}
\end{enumerate}

We will use positive Z as upwards direction
Looking at a bird's eye view of the field with our line on the bottom.
Positive X is towards right, and positive Y is going up. Our Vehicle will have a mass, m.

We are using a DJI Naza Lite
that takes as it's input ($F_{z}$ is also in the frame of the vehicle)
\begin{enumerate}
\item $F_{z}$
\item $\phi$
\item $\theta$
\item $\dot{\psi}$
\end{enumerate}

Our state that we will be estimating is (all states are in global frame)\\
\begin{center}
$
\vec{x} = 
\begin{Bmatrix}
x \\
y \\
z \\
\dot{x} \\
\dot{y} \\
\dot{z} \\
\phi \\
\theta \\
\psi \\
\end{Bmatrix}
=
\begin{Bmatrix}
x_{1} \\
x_{2} \\
x_{3} \\
x_{4} \\
x_{5} \\
x_{6} \\
x_{7} \\
x_{8} \\
x_{9} \\
\end{Bmatrix}
$
\end{center}

Our sensor measurement will be in the form (these are in vehicle frame)
\begin{center}
$
\vec{y} = 
\begin{Bmatrix}
\phi \\
\theta \\
\psi \\
\dot{x} \\
\dot{y} \\
z \\
\end{Bmatrix}
=
\begin{Bmatrix}
y_{1} \\
y_{2} \\
y_{3} \\
y_{4} \\
y_{5} \\
y_{6} \\
\end{Bmatrix}
$
\end{center}

our control input will be in the form (vehicle frame)
\begin{center}
$
\vec{u} = 
\begin{Bmatrix}
F_{z}\\
\phi \\
\theta \\
\psi \\
\end{Bmatrix}
=
\begin{Bmatrix}
u_{1} \\
u_{2} \\
u_{3} \\
u_{4}
\end{Bmatrix}
$
\end{center}

\section{Continuous Discrete Extended Kalman Filter}
We used the filter as described by Beard in his paper.
Here are the predict and update equations. Note: $\Delta t$ is time since last predict\\
Prediction:
\begin{enumerate}
\item $\hat{x} = \hat{x} + \Delta t f(\vec{x},\vec{u})$
\item $A = \frac{\Delta f}{\Delta x}$
\item $P = P + \Delta t (AP + PA^T + GQG^T)$
\end{enumerate}
Update
\begin{enumerate}
\item $C = \frac{\Delta c}{\Delta x}$
\item $L = PC^T(R + CPC^T)^{-1}$
\item $P = (I - LC)P$
\item $\hat{x} = \hat{x} + L(y - c(\vec{x}))$
\end{enumerate}

Since we don't have direct access to output, we are unable to change the roll, pitch, yaw in the prediction step
So we will being doing something unorthodox - just replacing roll pitch yaw with the MicroStrain roll pitch yaw every step.

\section{Prediction Model}
Our prediction model will continuous

\begin{center}
$\vec{\frac{dx}{dt}} =
f(\vec{x}, \vec{u}) =
\begin{Bmatrix}
x_{4}\\
x_{5}\\
x_{6}\\
-\frac{u_{1}}{m}\sin(x_{8})\\
\frac{u_{1}}{m}\sin(x_{7})\cos(x_{8})\\
\frac{u_{1}}{m}\cos(x_{7})\cos(x_{8})\\
0\\
0\\
u_{4}\\
\end{Bmatrix}
$
\end{center}

The linearized model of this around a state is (the 'A' matrix)
\begin{center}
$
\begin{Bmatrix}
\dot{\Delta x_{1}}\\
\dot{\Delta x_{2}}\\
\dot{\Delta x_{3}}\\
\dot{\Delta x_{4}}\\
\dot{\Delta x_{5}}\\
\dot{\Delta x_{6}}\\
\dot{\Delta x_{7}}\\
\dot{\Delta x_{8}}\\
\dot{\Delta x_{9}}\\
\end{Bmatrix}
=
\begin{Bmatrix}
\Delta x_{4}\\
\Delta x_{5}\\
\Delta x_{6}\\
-\frac{f_z}{m}\cos(x_8) \Delta x_8\\
-\frac{f_z}{m}\cos(x_7)\cos(x_8)\Delta x_7 -\frac{f_z}{m}\sin(x_7)\sin(x_8)\Delta x_8\\
-\frac{f_z}{m}\sin(x_7)\cos(x_8)\Delta x_7
-\frac{f_z}{m}\cos(x_7)\sin(x_8)\Delta x_8
\end{Bmatrix}
$
\end{center}

\section{Sensor Model}
Our sensor model is
\begin{center}
$
\hat{y} = c(\vec{x}) =
\begin{Bmatrix}
x_7 \\
x_8 \\
x_9 \\
x_4 \cos(x_9) \\
x_4 \sin(x_9) \\
\frac{x_3}{\cos(x_7)\cos(x_8)}
\end{Bmatrix}
$
\end{center}
The linearized model of this around a state (the C matrix)
\begin{center}
$
\begin{Bmatrix}
\Delta x_7\\
\Delta x_8\\
\Delta x_9\\
\cos(x_9)\Delta x_4 - x_4 \sin(x_9) \Delta x_9\\
\sin(x_9)\Delta x_4 + x_4 \cos(x_9) \Delta x_9\\
\frac{1}{\cos(x_7)cos(x_8)}\Delta x_3 +
\frac{x_3\sin(x_7)}{\cos(x_8)\cos^2(x_7)}\Delta x_7 +
\frac{x_3\sin(x_8)}{\cos(x_7)\cos^2(x_8)}\Delta x_8
\end{Bmatrix}
$
\end{center}

\end{document}